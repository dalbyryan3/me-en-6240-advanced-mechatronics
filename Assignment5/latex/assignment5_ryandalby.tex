\documentclass[12pt]{article}

% Packages 
\usepackage{amsmath}
\usepackage{datetime}
\usepackage{graphicx}
\usepackage{listings}

\graphicspath{{./images/}}

\newdate{date}{23}{02}{2022}
\title{
    Assignment 5 

    \large{
        ME EN 6240 Advanced Mechatronics
    }  
}
    
\author{
        Ryan Dalby
}
\date{\displaydate{date}}


\begin{document}
\maketitle

\section*{Exercise 3}
\verb|mathPIC.c| contains code for this problem.

\begin{itemize}
    \item[a.]
    Operations with jumps:
    \begin{itemize}
        \item
        \verb|j3=j1/j2|
        \item
        \verb|f3 = f1+f2;|
        \item
        \verb|f3 = f1-f2;|
        \item
        \verb|f3 = f1*f2;|
        \item
        \verb|f3 = f1/f2;|
        \item
        \verb|d3 = d1+d2;|
        \item
        \verb|d3 = d1-d2;|
        \item
        \verb|d3 = d1*d2;|
        \item
        \verb|d3 = d1/d2;|
    \end{itemize}

    Assembly code example with jump:
    \begin{verbatim}
        d3 = d1/d2;
    9d0031a8:	8fc40038 	lw	a0,56(s8)
    9d0031ac:	8fc5003c 	lw	a1,60(s8)
    9d0031b0:	8fc60040 	lw	a2,64(s8)
    9d0031b4:	8fc70044 	lw	a3,68(s8)
    9d0031b8:	0f400780 	jal	9d001e00 <__divdf3>
    9d0031bc:	00000000 	nop
    9d0031c0:	afc20060 	sw	v0,96(s8)
    9d0031c4:	afc30064 	sw	v1,100(s8)

    9d001e00 <__divdf3>:
    9d001e00:	00055040 000a5542 00054ac0 00047d42   
    9d001e10:	012f4825 000442c0 3c0e8000 012e4825   
    9d001e20:	00076840 000d6d42 000762c0 00067d42  
    9d001e30:	018f6025 00065ac0 018e6025 00a72826 
    9d001e40:	00ae2824 254fffff 2de107fe 102000cb
    9d001e50:	00000000                          
    \end{verbatim}

    \item[b.]
    For statements with no jumps, integer addition (and subtraction) results in the fewest assembly commands.

    \begin{verbatim}
        i3 = i1+i2;
    9d002fd4:	8fc30014 	lw	v1,20(s8)
    9d002fd8:	8fc20018 	lw	v0,24(s8)
    9d002fdc:	00621021 	addu	v0,v1,v0
    9d002fe0:	afc2004c 	sw	v0,76(s8)
    \end{verbatim}

    Char does not results in the least operations.
    This is because the PIC32 is a 32bit CPU and thus performs 32bit operations, for a char which is 8 bits extra 0s are added in front to make the char be able to operate with the 32bit CPU.
    The operation is then performed and the extra operations clear the unnecessary results from the padded calculation.
    The int datatype can perform directly without these padding/shifting commands.

    \item[c.]
    \begin{tabular}{|c|c|c|c|c|c|}
        \hline
        & char & int & long long & float & long double \\
        \hline
        + & 1.25(5) & 1.0(4) & 2.75(11) & 1.25(5)J&  2.0(8)J\\
        \hline
        - & 1.25(5) & 1.0(4) & 2.75(11) & 1.25(5)J&  2.0(8)J\\
        \hline
        * & 1.5(6) & 1.25(5) & 4.75 (19) & 1.25(5)J&  2.0(8)J\\
        \hline
        / & 1.75(7) & 1.75(7) & 3.0(12)J & 1.25(5)J&  2.0(8)J\\
        \hline
        
    \end{tabular}

    \item[d.]

    \begin{verbatim}

    __divdf3 is located at address 0x9d001e00 
    and takes up approximately 0x4b8=1208 bytes of memory.

    __divdi3 is located at address 0x9d0022b8 
    and takes up approximately 0x4b4=1204 bytes of memory.

    __subdf3 and __adddf3 are located at address 0x9d00276c 
    and take up approximately 0x430=1072 bytes of memory.

    __muldf3 is located at address 0x9d002b9c 
    and takes up approximately 0x32c=812 bytes of memory.

    __subsf3 and __addsf3 are located at address 0x9d003498 
    and take up approximately 0x278=632 bytes of memory.

    __divsf3 is located at address 0x9d003710 
    and takes up approximately 0x230=560 bytes of memory.

    __mulsf3 is located at address 0x9d003940 
    and takes up approximately 0x1bc=444 bytes of memory.

    \end{verbatim}

    \end{itemize}


\section*{Exercise 4}
\verb|bitManip.c| contains code for this problem.
\begin{itemize}
    \item 
    \verb|u3 = u1 & u2;|
    This uses 4 commands.

    \item 
    \verb<u3 = u1 | u2;<
    This uses 4 commands.

    \item 
    \verb|u3 = u2 << 4;|
    This uses 3 commands.

    \item 
    \verb|u3 = u1 >> 3;|
    This uses 3 commands.

\end{itemize}

\section*{Exercise 6}
\verb|mathFuncTest.c| contains code for this problem.
\begin{itemize}
    \item[a.]
    \begin{verbatim}
        Averaged over 10 calls to each function the execution time of each is:

        cosf(f1):
        2562 ns

        sqrtf(f1):
        3410 ns

        cos(d1):
        3460 ns

        sqrt(d1):
        4290 ns

    \end{verbatim}

    \item[b.]
    \begin{verbatim}
          f2 = cosf(f1);
        9d0028bc:	8fc40010 	lw	a0,16(s8)
        9d0028c0:	0f400fd5 	jal	9d003f54 <cosf>
        9d0028c4:	00000000 	nop
        9d0028c8:	afc20020 	sw	v0,32(s8)

          d2 = cos(d1);
        9d002b1c:	8fc40018 	lw	a0,24(s8)
        9d002b20:	8fc5001c 	lw	a1,28(s8)
        9d002b24:	0f40105e 	jal	9d004178 <__truncdfsf2>
        9d002b28:	00000000 	nop
        9d002b2c:	00402021 	move	a0,v0
        9d002b30:	0f400fd5 	jal	9d003f54 <cosf>
        9d002b34:	00000000 	nop
        9d002b38:	00402021 	move	a0,v0
        9d002b3c:	0f401217 	jal	9d00485c <__extendsfdf2>
        9d002b40:	00000000 	nop
        9d002b44:	afc20028 	sw	v0,40(s8)
        9d002b48:	afc3002c 	sw	v1,44(s8)
    \end{verbatim}
    Comparing cosf and cos we can see that cos can perform on an eight byte double floating point numbers and give a more accurate cosine estimation.
    The cosf routine takes less instructions than cos and in practice is a quicker execution as shown in a.
    The cosf routine also only performs with four byte floats and gives a less accurate estimation of cosine.
    The cos routine actually calls cosf after doing some bitwise manipulation.

    \item[c.]
    The libm.a file in 
    
    \begin{lstlisting}[breaklines]
    c:/program files (x86)/microchip/xc32/v2.30/bin/bin/../../lib/gcc/pic32mx/4.8.3/../../../../pic32mx/lib
    \end{lstlisting}

    was used by the linker.
\end{itemize}

\section*{Exercise 10}
% Global variables are stored at the beginning addresses of RAM, then next on "top" of that is the "heap", then on the very top is the "stack"
\verb|glob[5000]| occupies 20000 bytes in global memory, which assuming no memory on heap and 128KB = 131072 bytes of RAM, the largest size of an array of ints could take up 121072 bytes thus giving a max local int array of size 30268.

\end{document}